% \iffalse
% $Id: adobecaslon.dtx,v 1.4 2012-06-06 02:59:08 boris Exp $
%
% Copyright (c) 2012, Boris Veytsman
%
% All rights reserved.
%
% Redistribution and use in source and binary forms, with or without
% modification, are permitted provided that the following conditions
% are met: 
%
%    * Redistributions of source code must retain the above copyright
%    notice, this list of conditions and the following disclaimer. 
%    * Redistributions in binary form must reproduce the above
%    copyright notice, this list of conditions and the following
%    disclaimer in the documentation and/or other materials provided
%    with the distribution. 
%    * Neither the name of the original author nor the names of the
%    contributors may be used to endorse or promote products derived
%    from this software without specific prior written permission. 
%
% THIS SOFTWARE IS PROVIDED BY THE COPYRIGHT HOLDERS AND
% CONTRIBUTORS "AS IS" AND ANY EXPRESS OR IMPLIED WARRANTIES,
% INCLUDING, BUT NOT LIMITED TO, THE IMPLIED WARRANTIES OF
% MERCHANTABILITY AND FITNESS FOR A PARTICULAR PURPOSE ARE
% DISCLAIMED. IN NO EVENT SHALL THE COPYRIGHT OWNER OR CONTRIBUTORS
% BE LIABLE FOR ANY DIRECT, INDIRECT, INCIDENTAL, SPECIAL,
% EXEMPLARY, OR CONSEQUENTIAL DAMAGES (INCLUDING, BUT NOT LIMITED
% TO, PROCUREMENT OF SUBSTITUTE GOODS OR SERVICES; LOSS OF USE,
% DATA, OR PROFITS; OR BUSINESS INTERRUPTION) HOWEVER CAUSED AND ON
% ANY THEORY OF LIABILITY, WHETHER IN CONTRACT, STRICT LIABILITY,
% OR TORT (INCLUDING NEGLIGENCE OR OTHERWISE) ARISING IN ANY WAY
% OUT OF THE USE OF THIS SOFTWARE, EVEN IF ADVISED OF THE
% POSSIBILITY OF SUCH DAMAGE.
%
% \fi 
% \CheckSum{247}
%
%
%% \CharacterTable
%%  {Upper-case    \A\B\C\D\E\F\G\H\I\J\K\L\M\N\O\P\Q\R\S\T\U\V\W\X\Y\Z
%%   Lower-case    \a\b\c\d\e\f\g\h\i\j\k\l\m\n\o\p\q\r\s\t\u\v\w\x\y\z
%%   Digits        \0\1\2\3\4\5\6\7\8\9
%%   Exclamation   \!     Double quote  \"     Hash (number) \#
%%   Dollar        \$     Percent       \%     Ampersand     \&
%%   Acute accent  \'     Left paren    \(     Right paren   \)
%%   Asterisk      \*     Plus          \+     Comma         \,
%%   Minus         \-     Point         \.     Solidus       \/
%%   Colon         \:     Semicolon     \;     Less than     \<
%%   Equals        \=     Greater than  \>     Question mark \?
%%   Commercial at \@     Left bracket  \[     Backslash     \\
%%   Right bracket \]     Circumflex    \^     Underscore    \_
%%   Grave accent  \`     Left brace    \{     Vertical bar  \|
%%   Right brace   \}     Tilde         \~} 
%
%
% \MakeShortVerb{|}
% \GetFileInfo{adobecaslon.sty}
% \title{\LaTeX{} Support For Adobe Caslon Fonts}
% \author{Boris Veytsman\thanks{%
% \href{mailto:borisv@lk.net}{\texttt{borisv@lk.net}},
% \href{mailto:boris@varphi.com}{\texttt{boris@varphi.com}}}} 
% \date{\filedate, \fileversion}
% \maketitle
% \begin{abstract}
%   This package provides \LaTeX{} support for the Adobe Caslon
%   Fonts. It has
%   everything you need to typeset in this font, except the fonts
%   themselves.
% \end{abstract}
% \tableofcontents
%
% \clearpage
%
%
%\section{Introduction}
%\label{sec:intro}
%
% This package provides support files for the Adobe Caslon font family
% in pdfLaTeX. (The LuaTeX and XeTeX engines support system fonts
% directly, but there are still good reasons to use pdfTeX.)
%
%
%\section{Installation}
%\label{sec:install}
%
%
%
% The following instructions assume a TeX Directory Structure
% compliant TeX system. If you don't know what that is, you probably
% have one! Otherwise, you'll need to work out where to put the files
% yourself.
%
% First, you need to purchase the fonts themselves: the |pfb| files
% are \emph{not} included in the package.  If you got the fonts from
% Adobe, do not rename the |pfb| files.  However, if
% there are uppercase letters in your file names, downcase them.  If
% you obtained the files from another source, rename the fonts
% according to Table~\ref{tab:PFB}.  Install the files into
% \path{$TEXMF/fonts/type1/adobe/adobecaslon}.
%
% Then, download
% \url{http://ctan.tug.org/install/fonts/psfonts/adobe/adobecaslon.tds.zip}
% and unzip this file in \path{$TEXMF}.
%
% Finally, add the line |Map pac.map| to your |updmap.cfg| file, and
% run |updmap| and |texhash| to update the configuration files and
% file names database.
%
%
% \begin{table}[tp]
%   \centering
%   \caption{PFB Files}
%   \label{tab:PFB}
%
%   \begin{tabular}{ll}
%     \toprule
%    File  &  Font \\
%    \midrule
% |awrg____.pfb| & Adobe Caslon Regular\\
% |awi_____.pfb| & Adobe Caslon Italic \\
% |awsb____.pfb| & Adobe Caslon Semibold \\
% |awsbi___.pfb| & Adobe Caslon Semibold Italic \\
% |awb_____.pfb| & Adobe Caslon Bold\\
% |awbi____.pfb| & Adobe Caslon Bold Italic\\
%    \bottomrule
%   \end{tabular}
%
% \end{table}
%
%
% The style |adobecaslon.sty| provides a number of commands for using the
% font family (see Section~\ref{sec:adobecaslon.sty}).
%
%
%\section{Using \LaTeX{} Style }
%\label{sec:adobecaslon.sty}
%
% To use the package, add to the preamble of your document the usual
% incantation \cmd{\usepackage}\oarg{options}|{adobecaslon}|.
%
% By default, Adobe Caslon is made the default roman font (option
% |rmdefault|). The option |normdefault| prevents this.
% The option |scaled=...| scales Adobe fonts to the given scale.
%
% The package provides Adobe fonts in two shapes: upright and italic,
% and in three weights: medium (|m|), semibold (|sb|) and bold (|b|).
% The can be selected in the usual way, for example
% \begin{verbatim}
% \fontfamily{pac}\fontshape{it}\fontseries{sb}\selectfont
% \end{verbatim}
%
% \DescribeMacro{\adobecaslonfamily}
% \DescribeMacro{\textadobecaslon}
% Alternatively you can use a declaration |\adobecaslonfamily| and a
% command |\textadobecaslon| to set the family.  
%
% \DescribeMacro{\sbseries}
% \DescribeMacro{\textsb}
% You can use the standard \LaTeX{} commands to select the shape and
% weight of the font.  The package also provides a new declaration
% |\sbseries| and a command 
% |\textsb| modeled after the familiar commands |\bfseries| and
% |\textbf|, which select semi-bold weights.  
%
% \StopEventually{
%   \clearpage
%
%   \bibliography{adobecaslon}
%   \bibliographystyle{unsrt}}
%
% \clearpage
%\section{Implementation}
%\label{sec:impl}
%
%\subsection{Identification}
%\label{sec:ident}
%
% We start by declaring who we are.
%    \begin{macrocode}
%<style>\NeedsTeXFormat{LaTeX2e}
%<driver>\ProvidesFile{adobecaslon.dtx}
%<style>\ProvidesPackage{adobecaslon}
%<style>   [2012/06/05 v1.0 Using Adobe Caslon Font in LaTeX]
%    \end{macrocode}
% And the driver code:
%    \begin{macrocode}
%<*driver>
\documentclass{ltxdoc}
\usepackage{booktabs}
\usepackage{url}
\usepackage[tableposition=top]{caption}
\usepackage{hypdoc}
\usepackage[normdefault]{adobecaslon}
\PageIndex
\CodelineIndex
\RecordChanges
\EnableCrossrefs
\begin{document}
  \DocInput{adobecaslon.dtx}
\end{document}
%</driver>
%    \end{macrocode}
%
%
%
%\subsection{Fontinst Driver}
%\label{sec:pac-drv}
%
% This follows~\cite{fontinstallationguide}.
% 
% First, the preamble
%    \begin{macrocode}
%<*pac-drv>
\input fontinst.sty
\substitutesilent{bx}{b}
%    \end{macrocode}
%  
%
% Starting recording transforms:
%    \begin{macrocode}
\recordtransforms{pac-rec.tex}
%    \end{macrocode}
% A bunch of fonts:
%    \begin{macrocode}
\transformfont{pacr8r}{\reencodefont{8r}{\fromafm{awrg____}}}
\transformfont{pacri8r}{\reencodefont{8r}{\fromafm{awi_____}}}
\transformfont{pacs8r}{\reencodefont{8r}{\fromafm{awsb____}}}
\transformfont{pacsi8r}{\reencodefont{8r}{\fromafm{awsbi___}}}
\transformfont{pacb8r}{\reencodefont{8r}{\fromafm{awb_____}}}
\transformfont{pacbi8r}{\reencodefont{8r}{\fromafm{awbi____}}}
%    \end{macrocode}
%
% There is no hook in |fontinst.sty| for writing our own preamble to
% |.fd| file.  However, we need to add scaling commands to the
% preamble. OK, we will patch fontinst:
%    \begin{macrocode}
\fontinstcc
\def\fd_family#1#2#3{
   \a_toks{#3}
   \edef\lowercase_file{\lowercase{
     \edef\noexpand\lowercase_file{#1#2.fd}}}
   \lowercase_file
   \open_out{\lowercase_file}
   \out_line{\percent_char~Filename:~\lowercase_file}
   \out_line{\percent_char~Created~by:~tex~\jobname}
   \out_line{\percent_char~Created~using~fontinst~v\fontinstversion}
   \out_line{}
   \out_line{\percent_char~THIS~FILE~SHOULD~BE~PUT~IN~A~TEX~INPUTS~
      DIRECTORY}
   \out_line{}
   \out_line{\string\ProvidesFile{\lowercase_file}}
   \out_lline{[
      \the\year/
      \ifnum10>\month0\fi\the\month/
      \ifnum10>\day0\fi\the\day\space
      Fontinst~v\fontinstversion\space
      font~definitions~for~#1/#2.
   ]}
   \out_line{}
%    \end{macrocode}
% Here is our patch:
%    \begin{macrocode}
   \out_line{\string\expandafter\string\ifx\string\csname\space
     adobecaslon@scaled\string\endcsname\string\relax}
   \out_line{\space\string\let\string\adobecaslon@scaled\string\@empty}
   \out_line{\string\else}
   \out_line{\space\string\edef\string\adobecaslon@scaled\left_brace_char 
       s*[\string\csname\space adobecaslon@scaled\string\endcsname]
       \right_brace_char\percent_char}
   \out_line{\string\fi\percent_char}
   \out_line{}
%    \end{macrocode}
% End of the patch.
%    \begin{macrocode}
   \out_line{\string\DeclareFontFamily{#1}{#2}{\the\a_toks}}
   {
      \csname #1-#2\endcsname
      \out_line{}
      \let\do_shape=\substitute_shape
      \csname #1-#2\endcsname
      \let\do_shape=\remove_shape
      \csname #1-#2\endcsname
   }
   \x_cs\g_let{#1-#2}\x_relax
   \out_line{}
   \out_line{\string\endinput}
   \close_out{Font~definitions}
}
\normalcc
%    \end{macrocode}
% 
%
% Now we install the fonts.  First T1
%    \begin{macrocode}
\installfonts
\installfamily{T1}{pac}{}
\installfont{pacr8t}{pacr8r,newlatin}{t1}{T1}{pac}{m}{n}{
  <->\string\adobecaslon@scaled} 
\installfont{pacri8t}{pacri8r,newlatin}{t1}{T1}{pac}{m}{it}{
  <->\string\adobecaslon@scaled} 
\installfont{pacs8t}{pacs8r,newlatin}{t1}{T1}{pac}{sb}{n}{
  <->\string\adobecaslon@scaled} 
\installfont{pacsi8t}{pacsi8r,newlatin}{t1}{T1}{pac}{sb}{it}{
  <->\string\adobecaslon@scaled} 
\installfont{pacb8t}{pacb8r,newlatin}{t1}{T1}{pac}{b}{n}{
  <->\string\adobecaslon@scaled} 
\installfont{pacbi8t}{pacbi8r,newlatin}{t1}{T1}{pac}{b}{it}{
  <->\string\adobecaslon@scaled} 
\endinstallfonts
%    \end{macrocode}
%
%
% Then TS1
%    \begin{macrocode}
\installfonts
\installfamily{TS1}{pac}{}
\installfont{pacr8c}{pacr8r,textcomp}{ts1}{TS1}{pac}{m}{n}{
  <->\string\adobecaslon@scaled} 
\installfont{pacri8c}{pacri8r,textcomp}{ts1}{TS1}{pac}{m}{it}{
  <->\string\adobecaslon@scaled} 
\installfont{pacs8c}{pacs8r,textcomp}{ts1}{TS1}{pac}{sb}{n}{
  <->\string\adobecaslon@scaled} 
\installfont{pacsi8c}{pacsi8r,textcomp}{ts1}{TS1}{pac}{sb}{it}{
  <->\string\adobecaslon@scaled} 
\installfont{pacb8c}{pacb8r,textcomp}{ts1}{TS1}{pac}{b}{n}{
  <->\string\adobecaslon@scaled} 
\installfont{pacbi8c}{pacbi8r,textcomp}{ts1}{TS1}{pac}{b}{it}{
  <->\string\adobecaslon@scaled} 
\endinstallfonts
%    \end{macrocode}
% 
% 
% And  OT1
%    \begin{macrocode}
\installfonts
\installfamily{OT1}{pac}{}
\installfont{pacr7t}{pacr8r,newlatin}{ot1}{OT1}{pac}{m}{n}{
  <->\string\adobecaslon@scaled} 
\installfont{pacri7t}{pacri8r,newlatin}{ot1}{OT1}{pac}{m}{it}{
  <->\string\adobecaslon@scaled} 
\installfont{pacs7t}{pacs8r,newlatin}{ot1}{OT1}{pac}{sb}{n}{
  <->\string\adobecaslon@scaled} 
\installfont{pacsi7t}{pacsi8r,newlatin}{ot1}{OT1}{pac}{sb}{it}{
  <->\string\adobecaslon@scaled} 
\installfont{pacb7t}{pacb8r,newlatin}{ot1}{OT1}{pac}{b}{n}{
  <->\string\adobecaslon@scaled} 
\installfont{pacbi7t}{pacbi8r,newlatin}{ot1}{OT1}{pac}{b}{it}{
  <->\string\adobecaslon@scaled} 
\endinstallfonts
%    \end{macrocode}
%
%
% And the end:
%    \begin{macrocode}
\endrecordtransforms
\bye
%</pac-drv>
%    \end{macrocode}
%
%\subsection{Fontmap Generation}
%\label{sec:fontmap}
%
% This is a standard procedure~\cite{fontinstallationguide}
%    \begin{macrocode}
%<*pac-map>
\input finstmsc.sty
\resetstr{PSfontsuffix}{.pfb}
\adddriver{dvips}{pac.map}
\input pac-rec.tex
\donedrivers
\bye
%</pac-map>
%    \end{macrocode}
%
%
%
%\subsection{Style File}
%\label{sec:style}
%
%
% Declare the package options:
%    \begin{macrocode}
%<*style>
\RequirePackage{kvoptions}
\DeclareBoolOption{rmdefault}
\DeclareComplementaryOption{normdefault}{rmdefault}
\DeclareStringOption[1.00]{scaled}
\ProcessKeyvalOptions*
\ifadobecaslon@rmdefault
  \renewcommand{\rmdefault}{pac}
\fi
\DeclareVoidOption{semibold}{\def\bfdefault{sb}}
%    \end{macrocode}
% 
%
% Some new commands:
%    \begin{macrocode}
\DeclareRobustCommand\adobecaslonfamily{\fontfamily{pac}\selectfont}
\DeclareTextFontCommand{\textadobecaslon}{\adobecaslonfamily}
\DeclareRobustCommand\sbseries{\fontseries{sb}\selectfont}
\DeclareTextFontCommand{\textsb}{\sbseries}
%</style>
%    \end{macrocode}
% 
%
%\Finale
%\clearpage
%
%\PrintChanges
%\clearpage
%\PrintIndex
%
\endinput
